\chapter{DESCRIÇÃO DE PROJETO}
\section{PÚBLICO ALVO}
O público-alvo deste trabalho divide-se em dois âmbitos principais:
\begin{itemize}
	\item \textbf{Âmbito Técnico:} O projeto é direcionado a profissionais e pesquisadores da área de Engenharia Elétrica, com ênfase em sistemas embarcados e telecomunicações, que busquem soluções para a linearização de amplificadores e processamento de sinais em tempo real.
	
	\item \textbf{Âmbito Social e Ambiental:} De forma indireta, o trabalho beneficia a sociedade em geral por meio do desenvolvimento de tecnologias que aprimoram a infraestrutura de comunicação. A otimização do consumo de potência contribui diretamente para a sustentabilidade ambiental (redução do gasto energético), enquanto a melhoria na transmissão de sinais reflete em serviços de maior qualidade para setores essenciais como saúde (telemedicina), lazer e segurança.
\end{itemize}

\section{RECURSOS NECESSÁRIOS}
Para a implementação das etapas propostas no objetivo, seria necessário os seguintes recursos:
\begin{itemize}
	\item 1(um) computador (disponível no GICS)
	\item Software ISE na versão 14.7(disponível gratuitamente)
	\item 1(uma) placa de desenvolvimento para programação em VHDL (adquirido individualmente)
	\item Software Python (disponível gratuitamente)
	\item Software Cadence Virtuoso (Disponível no GICS)
	
\end{itemize}

\section{RESULTADOS FUNDAMENTAIS A SEREM ATINGIDOS}
Para a parte de resultados, o que se espera é obtermos os mesmos resultados apresentados no trabalhos \cite{oliveira2025otimizacao_emicro}, porém agora implementados em uma FPGA. Assim, será avaliado a eficácia do filtro com coeficientes otimizados, que poderão ser comparados com os resultados já obtidos anteriormente. Também espera-se conhecer a necessidade de hardware de um ASIC, o número de bits de precisão e o consumo de potência do sistema digital pós implementação.


\section{CONTRIBUIÇÃO ESPERADA PARA A ÊNFASE E IMPORTÂNCIA PARA A FORMAÇÃO DO AUTOR}
Como contribuição , este trabalho estabelece um método para minimizar o espalhamento espectral ocasionado pela saturação da envoltória do sinal. Tal abordagem permite que o Amplificador de Potência (PA) opere em sua região de maior eficiência energética, otimizando o desempenho do sistema.
Sob a ótica da formação acadêmica, o desenvolvimento deste Trabalho de Conclusão de Curso promove a consolidação dos conhecimentos técnicos adquiridos ao longo da graduação e em atividades de Iniciação Científica (IC). A execução do projeto integra áreas fundamentais como Processamento Digital de Sinais (PDS), Sinais e Sistemas, e programação em Python e VHDL, com foco direto em sistemas embarcados. Assim, o trabalho cumpre o papel de transpor conceitos teóricos para a implementação prática e experimental.

Adicionalmente, a pesquisa oferece uma relevante contribuição científica, não apenas por meio da formação de pessoal qualificado, mas também pela produção intelectual e potencial publicação de artigos científicos.

\section{METODOLOGIA}
A metodologia proposta para este trabalho divide-se em etapas sequenciais que abrangem desde o refinamento teórico e algorítmico até a validação em hardware, conforme as seguintes fases:

\subsection{Decomposição e Modelagem do Sistema}
Inicialmente, o sistema será segmentado em blocos funcionais para facilitar a análise individual de cada componente do método de redução de espalhamento espectral. Nesta etapa, será realizada a conversão de operações aritméticas complexas em representações de números reais equivalentes, visando a compatibilidade com a arquitetura da FPGA. 

\subsection{Otimização e Análise de Precisão}
Com o sistema dividido em blocos, será realizada uma análise de quantização para determinar o nível de precisão necessário (número de bits em ponto fixo). O objetivo é otimizar as operações para reduzir o consumo de recursos de hardware sem comprometer a integridade do sinal. As otimizações buscarão o equilíbrio entre o erro de truncamento e a complexidade lógica.



\subsection{Implementação em Hardware (VHDL)}
A fase de implementação consistirá na tradução dos modelos desenvolvidos em Python para a linguagem de descrição de hardware VHDL. Utilizando o software ISE 14.7, o sistema será sintetizado para a placa de desenvolvimento, onde a lógica será mapeada em blocos lógicos configuráveis.

\subsection{Métodos de Validação e Análise de Resultados}
Para verificar a eficácia da implementação, serão aplicados métodos de validação quantitativos e qualitativos:
\begin{itemize}
	\item \textbf{Métricas de Erro:} Serão calculados o \textit{Normalized Mean Square Error} (NMSE) e o \textit{Error Vector Magnitude} (EVM) para comparar a fidelidade do sinal processado na FPGA em relação ao modelo de referência.
	\item \textbf{Análise Espectral:} Será avaliada a Densidade Espectral de Potência (PSD) para mensurar a redução do espalhamento espectral e a eficiência do Amplificador de Potência (PA).
	\item \textbf{Métricas de Hardware:} O desempenho do sistema será avaliado quanto ao consumo de área (utilização de LUTs, flip-flops e multiplicadores) e estimativa de consumo de potência, utilizando as ferramentas de análise do software Cadence Virtuoso e ISE.
\end{itemize}

\newpage
\section{CRONOGRAMA}
\noindent
\begin{center}
	\begin{ganttchart}[
		expand chart=\textwidth,
		vgrid, % Linhas verticais
		hgrid, % Linhas horizontais
		x unit=5cm, % Largura de cada unidade (mês)
		y unit title=1cm,
		y unit chart=1cm,
		title label font=\bfseries\footnotesize,
		canvas/.style={draw=black!50, dotted},
		bar/.style={fill=blue!70}, % Cor das barras
		bar height=0.6,
		group right shift=0,
		group top shift=0.7,
		group height=.3,
		group peaks tip position=0,
		]{1}{10} % Define o intervalo 
		
		% Títulos das Colunas (Meses)
		\gantttitle{2026}{10} \\
		\gantttitle{Mar}{1} 
		\gantttitle{Abr}{1} 
		\gantttitle{Mai}{1} 
		\gantttitle{Jun}{1} 
		\gantttitle{Jul}{1} 
		\gantttitle{Ago}{1} 
		\gantttitle{Set}{1} 
		\gantttitle{Out}{1} 
		\gantttitle{Nov}{1} 
		\gantttitle{Dez}{1} \\
		
		% Elementos do Cronograma
		\ganttbar{Pesquisa Bibliográfica}{1}{3} \\
		\ganttbar[bar label font=\footnotesize\color{black}, 
		bar label node/.append style={align=right}] % Define o alinhamento
		{Conversão dos códigos em\\Python para VHDL}{2}{5} \\
		\ganttbar{Implementação prática}{6}{7} \\
		\ganttbar{Análise de Resultados}{7}{8} \\
		\ganttbar{Escrita da Monografia}{8}{9} \\
		\ganttmilestone{Entrega Final}{10} 
		
		% Conexões (opcional)
		\ganttlink{elem1}{elem2}
		
	\end{ganttchart}
\end{center}
