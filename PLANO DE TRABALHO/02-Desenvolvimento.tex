\chapter{DESCRIÇÃO DE PROJETO}
\section{PÚBLICO ALVO}
O trabalho estaria direcionado para prófissionais da áres de engenharia elétrica, estando especificado nas áreas de sistemas embarcados e telecomunicações.

\section{RECURSOS NECESSÁRIOS}
Para a implementação das etapas propostas no objetivo, seria necessário os seguintes recursos:
\begin{itemize}
	\item 1(um) computador
	\item Software ISE na versão 14.7(disponível gratuitamente)
	\item 1(uma) placa de desenvolvimento para programação em VHDL
\end{itemize}

\section{RESULTADOS FUNDAMENTAIS A SEREM ATINGIDOS}
Para a parte de resultados, o que se espera é obtermos os mesmos resultados apresentados no trabalhos xxx, porém agora implementados em uma FPGA. Assim, será avaliado o eficácio do filtro com coeficientes otimizados, que poderão ser comparados com os resultados já obtidos anteriormente.

\section{CONTRIBUIÇÃO ESPERADA PARA A ÊNFASE E IMPORTÂNCIA PARA A FORMAÇÃO DO AUTOR}
Como contribuição, o trabalho tem a premissa de desenvolver um método para minimizar o espalhamento espectral causado pela saturação da envoltória do sinal, contribuindo para que o PA trabalha na sua região mais eficiente. Quanto a parte da formação do autor, a implicação do desenvolvimento do trabalho de conclusão de curso envolve áreas como: processamento digital de sinais, sinais e sistemas e programação em python e VHDL, todos focados na aréa de sistemas embarcados, implementando na prática aquilo que foi visto de forma teórica no decorrer do curso.

\newpage
\section{CRONOGRAMA}
\noindent
\begin{center}
	\begin{ganttchart}[
		expand chart=\textwidth,
		vgrid, % Linhas verticais
		hgrid, % Linhas horizontais
		x unit=5cm, % Largura de cada unidade (mês)
		y unit title=1cm,
		y unit chart=1cm,
		title label font=\bfseries\footnotesize,
		canvas/.style={draw=black!50, dotted},
		bar/.style={fill=blue!70}, % Cor das barras
		bar height=0.6,
		group right shift=0,
		group top shift=0.7,
		group height=.3,
		group peaks tip position=0,
		]{1}{10} % Define o intervalo 
		
		% Títulos das Colunas (Meses)
		\gantttitle{2026}{10} \\
		\gantttitle{Mar}{1} 
		\gantttitle{Abr}{1} 
		\gantttitle{Mai}{1} 
		\gantttitle{Jun}{1} 
		\gantttitle{Jul}{1} 
		\gantttitle{Ago}{1} 
		\gantttitle{Set}{1} 
		\gantttitle{Out}{1} 
		\gantttitle{Nov}{1} 
		\gantttitle{Dez}{1} \\
		
		% Elementos do Cronograma
		\ganttbar{Pesquisa Bibliográfica}{1}{3} \\
		\ganttbar[bar label font=\footnotesize\color{black}, 
		bar label node/.append style={align=right}] % Define o alinhamento
		{Conversão dos códigos em\\Python para VHDL}{2}{5} \\
		\ganttbar{Implementação prática}{6}{7} \\
		\ganttbar{Análise de Resultados}{7}{8} \\
		\ganttbar{Escrita da Monografia}{8}{9} \\
		\ganttmilestone{Entrega Final}{10} 
		
		% Conexões (opcional)
		\ganttlink{elem1}{elem2}
		
	\end{ganttchart}
\end{center}
