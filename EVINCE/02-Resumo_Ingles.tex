The development of efficient systems for next-generation wireless communications faces increasingly complex challenges, especially regarding the simultaneous maximization of linearity and energy efficiency of transmitters. This scenario becomes even more challenging when dealing with concurrent multi-band transmissions, such as those found in modern wireless communication standards. To address the nonlinearities introduced by power amplifiers (PA), digital predistortion (DPD) has been widely adopted as a linearization technique. However, the energy efficiency of transmitters is still limited due to inherent imperfections in the inverse modeling process performed by DPD, especially at high output powers. With this, recent research has introduced DPD saturation, which allows achieving maximum efficiency in exchange for controlled distortion. Although DPD saturation enables high efficiency, in multi-band transmission scenarios, the high growth of the spectral floor becomes the main limitation of this technique. Thus, this work explores the use of Finite Impulse Response (FIR) filters for spectral floor reduction caused by DPD saturation in concurrent dual-band scenarios. The filter was synthesized for a cutoff frequency equal to twice the bandwidth, the windowing technique and order selection were determined seeking the smallest transition band. Tests were performed with Python for filter development, Cadence Virtuoso for signal integrability analysis and Octave for graphical analysis. The filter, implemented with Hamming windowing and order equal to 51 for channel 1 with IEEE 802.11n at 2.4 GHz, and order 81 for channel 2 with LTE at 3.5 GHz, was able to significantly reduce the spectral floor and reach the limits of the standards in the power spectral density (PSD) metric, without degradation of the error vector magnitude (EVM) and an increase of only 0.6 dB in the peak-to-average power ratio (PAPR).