\chapter{CONCLUSÃO}

Na transmissão de sinais, cada vez mais essencial na sociedade contemporânea, a obtenção de uma saída linear e eficiente configura-se como um objetivo de grande relevância. A implementação de um filtro FIR digital no sistema DPD + PA contribui significativamente para o aumento da eficiência, sobretudo quando associado a métodos de saturação do pré-distorcedor. Essa etapa adicional permite que o amplificador opere mais próximo da região de saturação, resultando em um aumento da potência média durante o funcionamento.

Os resultados apresentados evidenciam que a adição do filtro implica em uma redução do pico de potência média, ao compararmos a envoltória equivalente do sinal sem saturação com aquela obtida após o processo de filtragem. Observa-se ainda que, embora exista um acréscimo de aproximadamente 0,6 dB entre os sinais saturado e filtrado, essa diferença não exerce impacto significativo no desempenho final do sistema.

No que se refere ao EVM, os resultados indicam que não há degradação relevante na qualidade do sinal, mantendo-o próximo ao valor ideal. Mesmo considerando que as simulações foram conduzidas apenas com o janelamento Hanning, os resultados obtidos permitem concluir que a implementação do filtro FIR exerce um efeito positivo no desempenho global do sistema.