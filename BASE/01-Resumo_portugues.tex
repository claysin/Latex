O desenvolvimento de sistemas eficientes para comunicações sem fio de última geração enfrenta desafios cada vez mais complexos, especialmente no que diz respeito à maximização simultânea da linearidade e da eficiência energética dos transmissores. Esse cenário se torna ainda mais desafiador quando se trata de transmissões de bandas múltiplas concorrentes, como as encontradas em padrões modernos de comunicação sem fio. Para lidar com as não linearidades introduzidas pelos amplificadores de potência (PA), o pré-distorcedor digital (DPD) tem sido amplamente adotado como uma técnica de linearização. No entanto, a eficiência energética dos transmissores ainda é limitada devido às imperfeições inerentes ao processo de modelagem inversa realizado pelo DPD, especialmente em altas potências de saída. Com isso, pesquisas recentes introduziram a saturação do DPD, que permite atingir a eficiência máxima em troca de distorção controlada. Embora a saturação do DPD permita alta eficiência, em cenários de transmissão de bandas múltiplas, o alto crescimento do fundo espectral torna-se a principal limitação dessa técnica. Dessa forma, este trabalho explora o uso de filtros de Resposta ao Impulso Finita (FIR) para redução do fundo espectral causado pela saturação do DPD em cenários de banda dupla concorrente. A filtro foi sintetizado para frequência de corte igual ao dobro da largura de banda, a técnica de janelamento e escolha da ordem foram determinadas buscando a menor banda de transição. Testes foram realizados com Python para desenvolvimento do filtro, Cadence Virtuoso para análise de integrabilidade do sinal e o Octave para análise gráfica. O filtro, implementado com janelamento Hamming e ordem igual a 51 para o canal 1 com IEEE 802.11n a 2,4 GHz, e ordem 81 para o canal 2 com LTE a 3,5 GHz, foi capaz de reduzir significativamente o fundo espectral e chegar aos limites das normas na métrica de densidade espectral de potência (PSD), sem degradação da magnitude do vetor de erro (EVM) e aumento de apenas 0,6 dB na razão de pico de potência para potência média (PAPR).